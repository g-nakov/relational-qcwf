\documentclass[12pt,a4paper]{article}
\usepackage{amssymb}
\usepackage{amsmath}
\usepackage{amsthm}
\usepackage{xspace}
\usepackage{tikz}
\usepackage{tikz-cd}
\usepackage{enumitem}
\usepackage{adjustbox}
\usepackage{stmaryrd}
\usepackage{mathrsfs}  
\usepackage{proof}
\usepackage[left=5em, right=5em]{geometry}

%%%%%%%%%%%%%%%%%%%%%%%%%%%%%%%%%%%% Link colors %%%%%%%%%%%%%%%%%%%%%%%%%%%%%%%%%%%
\usepackage{hyperref}
\usepackage[capitalize,nameinlink]{cleveref}

\hypersetup{
  colorlinks,
  citecolor=black,
  filecolor=black,
  linkcolor=black,
  urlcolor=black,
  linktoc=all
}

%%%%%%%%%%%%%%%%%%%%%%%%%%%%%%%%%%% Bibliography %%%%%%%%%%%%%%%%%%%%%%%%%%%%%%%%%%%

\usepackage[backend=biber, isbn=false, doi=false, style=numeric, 
citestyle=numeric, giveninits=true]{biblatex}
\bibliography{lib.bib}

%%%%%%%%%%%%%%%%%%%%%%%%%%%%%%%% Theorem enivornment %%%%%%%%%%%%%%%%%%%%%%%%%%%%%%%

%use continious uniform numbering

\newtheorem{thm}{Theorem}[section]
\newtheorem{prop}[thm]{Proposition}
\newtheorem{cor}[thm]{Corollary}
\newtheorem{lemma}[thm]{Lemma}
\newtheorem*{claim}{Claim}
\newtheorem*{fact}{Fact}

\theoremstyle{definition}
\newtheorem{remark}[thm]{Remark}
\newtheorem{definition}[thm]{Definition}
%% remove section prefix
\renewcommand{\thethm}{\arabic{thm}}
% \makeatletter
%\def\th@definition{%
%  \thm@notefont{\normalfont\itshape}
%}
%\makeatother


%%%%%%%%%%%%%%%%%%%%%%%%%%%%%%%%%%%% Always math %%%%%%%%%%%%%%%%%%%%%%%%%%%%%%%%%%%

\makeatletter
\let\ea\expandafter

\def\mdef#1#2{\ea\ea\ea\gdef\ea\ea\noexpand#1\ea{\ea\ensuremath\ea{#2}\xspace}}
\def\alwaysmath#1{\ea\ea\ea\global\ea\ea\ea\let\ea\ea\csname your@#1\endcsname\csname #1\endcsname
  \ea\def\csname #1\endcsname{\ensuremath{\csname your@#1\endcsname}\xspace}}

\makeatother

%%%%%%%%%%%%%%%%%%%%%%%%%%%%%%%% Letter decorations %%%%%%%%%%%%%%%%%%%%%%%%%%%%%%%%

\makeatletter

\newcount\foreachcount

\def\foreachletter#1#2#3{\foreachcount=#1
  \ea\loop\ea\ea\ea#3\@alph\foreachcount
  \advance\foreachcount by 1
  \ifnum\foreachcount<#2\repeat}

\def\foreachLetter#1#2#3{\foreachcount=#1
  \ea\loop\ea\ea\ea#3\@Alph\foreachcount
  \advance\foreachcount by 1
  \ifnum\foreachcount<#2\repeat}

% Particular commands for typefaces, sometimes with the first letter
% different.
\def\autofmt@n#1\autofmt@end{\mathrm{#1}}
\def\autofmt@b#1\autofmt@end{\mathbf{#1}}
\def\autofmt@d#1#2\autofmt@end{\mathbb{#1}\mathsf{#2}}
\def\autofmt@c#1#2\autofmt@end{\mathcal{#1}\mathit{#2}}
\def\autofmt@s#1#2\autofmt@end{\mathscr{#1}\mathit{#2}}
\def\autofmt@f#1\autofmt@end{\mathsf{#1}}
%\def\autofmt@k#1\autofmt@end{\mathfrak{#1}}
%% Particular commands for decorations.
\def\autofmt@u#1\autofmt@end{\underline{\smash{\mathsf{#1}}}}
\def\autofmt@U#1\autofmt@end{\underline{\underline{\smash{\mathsf{#1}}}}}
\def\autofmt@h#1\autofmt@end{\widehat{#1}}
\def\autofmt@r#1\autofmt@end{\overline{#1}}
\def\autofmt@t#1\autofmt@end{\widetilde{#1}}
\def\autofmt@k#1\autofmt@end{\check{#1}}

% Defining multi-letter commands.  Use this like so:
% \autodefs{\bSet\cCat\cCAT\kBicat\lProf}
\def\auto@drop#1{}
\def\autodef#1{\ea\ea\ea\@autodef\ea\ea\ea#1\ea\auto@drop\string#1\autodef@end}
\def\@autodef#1#2#3\autodef@end{%
  \ea\def\ea#1\ea{\ea\ensuremath\ea{\csname autofmt@#2\endcsname#3\autofmt@end}\xspace}}
\def\autodefs@end{blarg!}
\def\autodefs#1{\@autodefs#1\autodefs@end}
\def\@autodefs#1{\ifx#1\autodefs@end%
  \def\autodefs@next{}%
  \else%
  \def\autodefs@next{\autodef#1\@autodefs}%
  \fi\autodefs@next}

\makeatother

%%%%%%%%%%%%%%%%%%%%%%%%%%%%%%%%%%% Miscellaneous %%%%%%%%%%%%%%%%%%%%%%%%%%%%%%%%%%

\setlength{\parindent}{0pt}

\def\define{:=}

\let\comp\circ

\def\pow{\mathcal{P}}
\def\nat{\mathbb{N}}
\newcommand{\carr}[1]{\ensuremath{|#1|}}
\newcommand{\ob}[1]{\ensuremath{Ob(#1)}}

\renewcommand{\phi}{\varphi}
\alwaysmath{Gamma}


\newcommand{\terminal}[1]{\mathbf{1_{#1}}}
\newcommand{\of}[1]{\overset{#1}{:}}
\newcommand{\ofto}[1]{\overset{#1}{\to}}

\newcommand{\emptyCtxt}{\diamond}
\newcommand{\letin}[3]{\text{ let } (#1) = #2 \text{ in } #3 }

\def\definecat#1{\ea\gdef\csname c#1\endcsname{\ensuremath{\mathcal{#1}}\xspace}}
\foreachLetter{1}{27}{\definecat}

\def\definealg#1{\ea\gdef\csname alg#1\endcsname{\ensuremath{\mathscr{#1}}\xspace}}
\foreachLetter{1}{27}{\definealg}

%%%%%%%%%%%%%%%%%%%%%%%%%%%%%%%%%%% Comments %%%%%%%%%%%%%%%%%%%%%%%%%%%%%%%%%%%%%%%

\newcommand{\FNF}[1]{\textcolor{blue}{\textbf{FNF:} #1}}
\newcommand{\GN}[1]{\textcolor{magenta}{\textbf{GN:} #1}}
\newcommand{\comment}[1]{\textcolor{gray}{\textit{#1}}}


\def\definecat#1{\ea\gdef\csname c#1\endcsname{\ensuremath{\mathcal{#1}}\xspace}}
\foreachLetter{1}{27}{\definecat}
\def\definealg#1{\ea\gdef\csname alg#1\endcsname{\ensuremath{\mathscr{#1}}\xspace}}
\foreachLetter{1}{27}{\definealg}

\autodefs{\cCat\cSet\cRGph\cSet\cORGph\bT\bU\dM}

\def\cAsm{\mathcal{A}\mathit{sm}^*(\algA)}
\def\cFamORG{\mathcal{F}\mathit{am-}\mathcal{ORG}}


\def\src{_{src}}
\def\rfl{_{refl}}
\def\tgt{_{tgt}}

\renewcommand{\O}{_{O}}\alwaysmath{O}
\newcommand{\R}{\ensuremath{_{R}}}

\def\Assembly{Assembly$^*$\xspace}
\def\Assemblies{Assemblies$^*$\xspace}
\def\assembly{assembly$^*$\xspace}
\def\assemblies{assemblies$^*$\xspace}

\def\Asmrg{Reflexive graph with realisable objects\xspace}
\def\Asmrgs{Reflexive graphs with realisable objects\xspace}
\def\asmrg{reflexive graph with realisable objects\xspace}
\def\asmrgs{reflexive graphs with realisable objects\xspace}

%%%%%%%%%%%%%%%%%%%%%%%%%%%%%%%%%%%%%%%%%%%%%%%%%%%%%%%%%%%%%%%%%%%%%%%%%%%%%%%%%%%%


\title{\vspace{-5em} Relational realisability model for QTT}
\date{\vspace{-3em}}

\begin{document}
\maketitle

Our aim is to build a concrete realisability model for QTT which allows for meta-reasoning with results derived from parametricity. To that end, a construction of a quantitative category with families (QCwF) is presented \cite{Atkey2018}. We follow the relational approach to types introduced by Reynolds for typed lambda calculus \cite{Ma1992} and later refined for dependent types theories \cite{Atkey2014a}.\\

Once and for all fix a usage semiring $R$ and an $R$-linear combinatory algebra \algA\footnote{In case some non-trivial properties of $\algA$ are required, we will assume that $\algA$ is a graph model (see \cite{Engeler1981})}.

\subsection*{Taking stocks}
\begin{definition}[\Assemblies]
  An \assembly \Gamma is a pair $(|\Gamma|, e)$ where $|\Gamma|$ is a carrier set and  $e$ is a realisability function $|\Gamma| \to \pow(\algA)$.
\end{definition}
Given some $\gamma \in |\Gamma|$, $e(\gamma)$ is interpreted as the set of witnesses for the existence of $\gamma$. To emphasize on that aspect, we write $a \vDash_\Gamma \gamma$ to denote $a \in e(\gamma)$. Moreover, let $\lfloor\Gamma\rfloor$ stand for the set of realisable elements - $\{ \gamma \in |\Gamma| : e_\Gamma(\gamma) \neq \emptyset \} \subseteq |\Gamma|$. \\

\textbf{A morphism between two \assemblies $\Gamma$ and $\Delta$} is a function $f : \lfloor \Gamma \rfloor  \to |\Delta|$ that is realisable - there exists a realiser $a_f \in \algA$ that tracks the function $f$ in the following sense:
\begin{center}
  for every $\gamma$ in $dom(f)$ and $a_\gamma$ in $\algA$,  $a_\gamma \vDash_\Gamma \gamma \implies a_f . a_\gamma \vDash_\Delta f(\gamma)$ holds.
\end{center}

Note that multiple realisers for the same function $f$ do not induce multiple morphisms. Let $\widehat{f} : |\Gamma| \to |\Delta|$ designate an extension of $f$ over $|\Gamma|$. Often instead of the \assembly morphism $f : X \to Y$, we may provide directly $\widehat{f}$ alongside a realiser for $f$.\\

Using these notions we can construct a category $\mathbf{\cAsm}$.

\begin{definition}[Reflexive graph]
  \label{def:refl-graph}
  A reflexive graph (r.g.) $G$ is a tuple $(G\O, G\R, G\rfl, G\src, G\tgt)$, where $G_O$ and $G_R$ are sets, $G\src : G\R \to G\O$, $G\tgt: G\R \to G\O$ and $G\rfl : G\O \to G\R$ are functions, s.t.\ the identities hold: $$G\src \comp G\rfl = id_{G\O} = G\tgt \comp G\rfl$$
\end{definition}

$G\O$ and $G\R$ stand for a set of objects and a set of relations, respectively. In general, reflexive graphs are less structured than categories as relations need not compose.\\

\textbf{A morphism between reflexive graphs $G$ and $H$} is a pair of functions $({f_o : G_O \to H_O},$ ${f_r : G_R \to H_R})$, s.t.\ all of the depicted squares commute:
\begin{center}
\adjustbox{scale=1.2}{
    % https://tikzcd.yichuanshen.de/#N4Igdg9gJgpgziAXAbVABwnAlgFyxMJZABgBpiBdUkANwEMAbAVxiRAAoAfADQH0B5TqQAEAHVEA1KHTgALXsD7EAvgEoQy0uky58hFGQBMVWoxZsufAEpCxomgBEZ8xbytqNWkBmx4CRQ3ITemZWRA5OAE0BW3FHZwVo-g9NbV89ANJjahDzcK5omxE4pzlEtxSTGCgAc3giUAAzACcIAFskAEZqHAgkMhAAIxgwKCQAZgHcsJBucThmgGMQagY6YYYABR0-fRAGGEacTybWjsQB3q6csxm50WbGhhX99ZgtnYzwrDBsWBe5Fgjv1UiAWu1riArhdqMNRkgALSTG6hNj3HA1Y6rN4fdL+cLNLA1WRYkCA4EXUHg86BKF9RDjFF5ECRcSPZ7YjbbPF7H5-VhUs4THr02lwsYMqa3Nis0QLZaCiEMkVIMUjCVIqWo8KyjGktZcz74kCE4nHRXnbp04WmbVg3jNE5goUw62IWnTNiNXh9ZQUZRAA
  \begin{tikzcd}
  {G_O} \arrow[dd, "G\rfl" description] \arrow[rr, "f_o"]                                 &  & {H_O} \arrow[dd, "H\rfl" description]                                 \\
  &  &                                                                                         \\
  {G_R} \arrow[uu, "G\src", bend left] \arrow[uu, "G\tgt"', bend right] \arrow[rr, "f_r"] &  & {H_R} \arrow[uu, "H\src", bend left] \arrow[uu, "H\tgt"', bend right]
  \end{tikzcd}
}
\end{center}

Reflexive graphs equipped with r.g.\ morphisms form \textbf{a category $\cRGph$}. The terminal object $\terminal\cRGph$ is $(\{\star\}, \{\star\}, id, id, id)$. \\

We use reflexive graphs to give a dyadic interpretation of types in the spirit of \cite{Ma1992}.


\subsection*{Augmenting reflexive graphs with realisability information}

One could easily generalize reflexive graphs by considering object and relation components from arbitrary categories instead of $\cSet$. As our purpose is to build a relational model capable of exploiting realisability, we replace the set of objects with an \assembly and retain the $\cSet$-based representation of relations.

\begin{definition}[\Asmrg]
  \footnote{\FNF{It's not easy, but we should think of a good name for this --- ``reflexive graph of $X$'' usually means ``reflexive graph in the category of $X$s'', i.e.\ where $G\O$ and $G\R$ are $X$s, and the maps are $X$ morphisms.}}
  A \asmrg $G$ is a tuple $(G\O, G\R, G\rfl, G\src, G\tgt)$, where $G\O \in \ob\cAsm$, $G\R$ is a set and the functions $G\rfl: |G\O| \to G\R, \; G\src : G\R \to |G\O|, \; G\tgt : G\R \to |G\O|$ are such that the identities in \cref{def:refl-graph} are satisfied.
\end{definition}

Accordingly, \textbf{a morphism between two \asmrgs $G$ and $H$} is a pair $(f\O: G\O \to H\O, f\R : G\R \to H\R)$, s.t.\ there exists an extension $\widehat{f\O}$, s.t.\ $(\widehat{f\O}, f\R)$ is an $\cRGph$ morphism between the reflexive graphs $(|G\O|, G\R, G\rfl, G\src, G\tgt)$ and $(|H\O|, H\R, H\rfl, H\src, H\tgt)$.\\

With these components, we obtain a category $\cORGph(\cAsm)$. By considering \asmrgs of shape $(X,|X|,id_X,id_X,id_X)$, we identify an isomorphic copy of $\cAsm$ inside $\cORGph(\cAsm)$.\\


A terminal object $\mathbf{1}_{\cORGph(\cAsm)}$ in $\cORGph(\cAsm)$ is a tuple $(\mathbf{1}_{\cAsm}, \{*\}, id, id, id)$, where $\mathbf{1}_{\cAsm}$ is the terminal \assembly $(\{\star\},f)$, with $f$ defined as $\star \mapsto \{I\}$.

\begin{definition}[Family of \asmrgs]
  \label{def:fam-refl-graphs}
  \footnote{\FNF{ditto here, namingwise}}
  Let $\cC$ be a category with a terminal object. Given a reflexive graph $\Gamma \in \ob\cC$, a family of \asmrgs over $\Gamma$ is a tuple $S\define(S_O, S_R, S\rfl, S\src, S\tgt)$, where:
  \begin{itemize}[noitemsep]
    \item $S\O : \Gamma\O \to \cAsm$
    \item $S\R : \Gamma\R \to \cSet$
    \item a $\Gamma$-indexed collection of functions $S\rfl \define \{f_\gamma : |{S\O(\gamma)}| \to S\R(\Gamma\rfl(\gamma)) \}_{\gamma \in \Gamma\O}$
    \item $S\src \define \{f_\gamma : S\R(\gamma) \to |S\O(\Gamma\src(\gamma))| \}_{\gamma \in \Gamma\R}$
    \item $S\tgt \define \{f_\gamma : S\R(\gamma) \to |S\O(\Gamma\tgt(\gamma))| \}_{\gamma \in \Gamma\R}$ 
  \end{itemize}
  such that 
  \begin{itemize}[noitemsep]
    \item each identity in the following collection is satisfied:
     $$S_\sigma(\Gamma\rfl(\gamma)) \comp S\rfl(\gamma) = id \text{ for every } \gamma \in \Gamma\O, \sigma \in \{src, tgt \}$$
  \end{itemize}

\end{definition}

\textbf{A morphism $M$ between two families $S$ and $T$ of \asmrgs over $\Gamma$} is a pair $(M\O, M\R)$ of $\Gamma$-indexed collections of morphisms:
\begin{itemize}[noitemsep]
  \item $M\O \define \{f_\gamma : S\O(\gamma) \to T\O(\gamma) \}_{\gamma \in \Gamma\O}$ \footnote{\FNF{should be tracked?}}
  \item $M\R \define \{f_\gamma : S\R(\gamma) \to T\R(\gamma) \}_{\gamma \in \Gamma\R}$
\end{itemize}
such that there exists a collection of some fixed extensions $\widehat{M\O}\define \{\widehat{f_\gamma}\}_{\gamma \in \Gamma\O}$, for which the following identities are satisfied:
\begin{itemize}[noitemsep]
  \item $T\rfl(\widehat{M\O}(\gamma)(s_o)) = M\R(\Gamma\rfl(\gamma))(S\rfl(\gamma)(s_o))$ for every $\gamma \in \Gamma\O$, $s_o \in |S\O(\gamma)|$
  
  \item $T\src(M\R (\gamma)(s_r)) = \widehat{M\O}(\Gamma\src(\gamma)) (S\src(\gamma)(s_r))$ for every $\gamma \in \Gamma\R$, $s_r \in S\R(\gamma)$
  
  \item $T\tgt(M\R (\gamma)(s_r)) = \widehat{M\O}(\Gamma\tgt(\gamma))(S\tgt(\gamma)(s_r))$ for every $\gamma \in \Gamma\R$, $s_r \in S\R(\gamma)$
\end{itemize}
Families of \asmrgs over $\Gamma$ and their morphisms forms a category $\cFamORG(\cAsm, \Gamma)$.\\

The terminal object $\terminal{\cFamORG(\cAsm, \Gamma)}$ is given by $(\lambda \gamma_r. \terminal\cAsm, \lambda \gamma_r. \{*\}, \terminal{refl}, \terminal{tgt}, \terminal{src})$, where $\terminal{\sigma}$ are the appropriate functions.

%A terminal family of \asmrg over $\Gamma$, $1_\Gamma$, consists of two constant functions, mapping $\gamma\in\Gamma$ to a terminal \assembly $1$, and three $\Gamma$-indexed collections with a sole element $id_1$.

%There is an apparent discrepancy in definition of family of r.g of assemblies and morphisms between them - morphisms ignore any realisability information. This is due to the role they  play in interpretation of QTT - semantic terms in the $0$-fragment of the theory are interpreted as morphisms from the terminal family and hence do not require any realizers.\\
%It is then tempting to continue discarding realisability information in \cref{def:fam-refl-graphs}. However, as families of r.g of assemblies will model the semantic types in QCwF, the approach seems infeasible.
%\begin{itemize}[noitemsep]
%  \item types provide realisability blueprint for resourced semantic terms. Thus we cannot substitute \cSet for \cAsm(\algA).
%  \item given a context $\Gamma \in \ob\cRGph$, $\rho \in R$ and a type $S \in Ty(\Gamma)$, the resourced context extension $\Gamma.\rho S \in \ob{\cRGph(\cAsm(\algA))}$ makes use of the trackers for $src, tgt$ and $refl$ morphisms. (Note that these are not needed when building an ordinary CwF).
%\end{itemize}

\subsection*{A CwF from families of \asmrgs}
As a first step toward obtaining a relationally parametric realisability model of QTT, we construct a concrete CwF using families of \asmrgs.\\

Consider the category $\cRGph$ with terminal object $\terminal{\cRGph}$.\\
Let $\Gamma, \Delta \in \ob\cRGph$, define: 
\begin{itemize}[noitemsep]
  \item the collection of semantic types $Ty(\Gamma)$ as the collection of families of \asmrg over $\Gamma$.
  
  \item given a type $S \in Ty(\Gamma)$, an element $M \in Tm(\Gamma, S)$ is a pair of functions\\ $({M\O : \forall \gamma \in \Gamma\O. |S\O(\gamma)|,\, M\R : \forall \gamma \in \Gamma\R. S\R(\gamma)})$, s.t. 
  \begin{align*}
    \forall \gamma \in \Gamma\O. {S\rfl} (M\O(\gamma)) &= M\R(\Gamma\rfl(\gamma))\\
    \forall \gamma \in \Gamma\R. {S\src} (M\R(\gamma)) &= M\O(\Gamma\src(\gamma))\\
    \forall \gamma \in \Gamma\R. {S\tgt} (M\R(\gamma)) &= M\O(\Gamma\tgt(\gamma))
  \end{align*}
  
  \item given $f : \Gamma \to \Delta$, substitutions in types and terms is a precomposition with f on the object and relation components of types and terms respectively. Clearly, these operations are compatible with identity and composition in $\cRGph.$
  
  \item context extension: Suppose $S \in Ty(\Gamma)$, construct a r.g. $\Gamma.S$ as :
  \begin{align*}
    (\Gamma.S)\O &= \{(\gamma, x) : \gamma \in \Gamma\O, x \in \carr{S\O(\gamma)}\}\\
    (\Gamma.S)\R &= \{(\gamma, r) : \gamma \in \Gamma\R, r \in S\R(\gamma)\}\\
    (\Gamma.S)\rfl(\gamma, x) &= (\Gamma\rfl(\gamma), S\rfl(\gamma)(x))\\
    (\Gamma.S)_\sigma(\gamma, r) &= (\Gamma_\sigma(\gamma), S_\sigma(\gamma)(r)),\quad \sigma \in \{src, tgt\}
  \end{align*}
  \begin{claim}
    $Hom_{\cRGph}(\Delta, \Gamma.S) \cong \{(f,M): f : \Delta \to \Gamma, M \in Tm(\Delta, S\{f\}\}$, natural in $\Delta$.
  \end{claim}
  
%  Given a morphism $(g\O, g\R) : \Delta \to \Gamma.S$, we have that $(\Gamma . S)\rfl \comp g\O  = g\R \comp \Delta\rfl $. Unwinding the definitions for some $\delta \in \Delta$, $(\Gamma\rfl( \pi_O(g\O(\delta))), S\rfl(\pi_0(g\O\delta))(\pi_1(g\O(\delta))) )$

\end{itemize}

\subsection*{Upgrading to a QCwF}
Recall the definition of a QCwF from \cite{Atkey2018}.
Given a usage semiring $R$, a $R$-QCwF consists of:
\begin{enumerate}[noitemsep]
  \item A CwF $(\cC, 1, Ty, Tm, -.-, \langle-.-\rangle)$
  
  \item A category $\cL$ with a faithful functor $U : \cL \to \cC$
  
  \item A functor $(+) : \cL \times_\cC \cL \to \cL$, s.t $U(\Gamma_1 + \Gamma_2) = U(\Gamma_1) = U(\Gamma_2)$\footnote{the second equality being trivially satisfied}. $\cL \times_\cC \cL$ denotes the pullback $\cL \xrightarrow{U} \cC \xleftarrow{U} \cL$.\\
  Additionally, there exists an object $\lozenge \in \cL$, s.t. $U\lozenge = 1$.
  
  \item A functor $\rho(-): \cL \to \cL$ for each $\rho \in R$, s.t $U(\rho(-)) = U(-)$.
  
  \item A collection $RTm(\Gamma, S)$ for each $\Gamma \in \cL$ and $S \in Ty(U\Gamma)$, equipped with an injective function $U_{\Gamma,S} : RTm(\Gamma, S) \to Tm(U\Gamma, S)$.\\
  For an $\cL$ morphisms $f: \Gamma \to \Delta$ and types $S \in Ty(U\Gamma)$, a function $-\{f\} : RTm(\Delta, S) \to RTm(\Gamma, S\{f\})$, s. t. $U(-\{f\}) = (U(-))\{Uf\}$.
         
  \item Given $\Gamma \in \cL$, $\rho \in R$ and $S \in Ty(U\Gamma)$, an object $\Gamma. \rho S$, s.t $U(\Gamma. \rho S) = U\Gamma.S$.\\
  Additionally, there exist the following natural transformations:
  \subitem $emp_\pi : \lozenge \to \pi\lozenge$\footnote{..., s.t. $U(emp_\pi)=id_1$ \FNF{since 1 terminal, this is automatic?}}
  \subitem $emp_+ : \lozenge \to \lozenge + \lozenge$\footnote{..., s.t. $U(emp_+)=id_1$ \FNF{ditto}}
  \subitem $ext_\pi : \pi\Gamma. (\pi\rho S) \to \pi(\Gamma. \rho S)$, s.t. $U(ext_\pi) = id$ 
  \subitem $ext_+ : (\Gamma_1 + \Gamma_2).(\rho_1 + \rho_2)S \to \Gamma_1.\rho_1 S + \Gamma_2 . \rho_2 S$, s.t.  $U(ext_+) = id$ 
  
  \item Given $\Gamma \in \cL$, $S \in Ty(U\Gamma)$, there exists \label{def:qcwf-morphisms}:
  \subitem a morphism $p_{\Gamma.S} : \Gamma. 0S \to \Gamma$, s.t. $U(p_{\Gamma.S})=p_{U\Gamma.S}$
  \subitem an element $v_{\Gamma.S} \in RTm(0\Gamma.1S, S\{p_{U\Gamma.S}\})$, s.t. $U(v_{\Gamma.S})=v_{U\Gamma.S}$
  \subitem a morphism $wk(f, \rho S') : \Gamma.\rho S'\{Uf\} \to \Delta.\rho S'$ for each $f : \Gamma \to \Delta$, $S' \in Ty(U\Gamma, \Delta)$ s.t. $U(wk(f, \rho S'))=wk(Uf, S')$
  \subitem let $\Gamma_1, \Gamma_2 \in \cL$, s.t $U\Gamma_1 = U\Gamma_2$ and $M \in RTm(\Gamma_2, S)$. There is a morphism $\overline{\rho M} : \Gamma_1 + \rho \Gamma_2 \to \Gamma_1 . \rho S$, s.t $U(\overline{\rho M})= \overline{UM}$
  \subitem a morphism $\overline{M} : \Gamma \to \Gamma.0S$ for $M \in Tm(U\Gamma, S)$, s.t. $U(\overline{M})=\overline{M}$.
\end{enumerate}

Now to construct the concrete QCwF, consider the CwF from previous section as the underlying category.\\

Let $\cL := \cORGph(\cAsm)$, the category of \asmrgs\footnote{\FNF{``half-assemblied'' r.g.s (yes, we also need notation for them)}} and let $U$ be the functor $\cORGph(\cAsm) \to \cRGph$, sending the object \assembly to its underlying set, forgetting the realisability relation.\\

For the addition structure, let $\Gamma', \Gamma''$ be \asmrgs, s.t  $|\Gamma'\O|=|\Gamma''\O|$ and $\Gamma'\R=\Gamma''\R$ \footnote{\FNF{no \carr{\cdot} for $\R$ part}}. Construct the r.g. of assemblies $\Gamma \define \Gamma' + \Gamma''$, where:
\begin{itemize}
    \item $\Gamma\O \define (\carr{\Gamma'\O}, \vDash_\Gamma)$ with $ a \vDash_\Gamma \gamma$ iff there exist $x,y \in \algA$, s.t. $a = [x,y]$ and $x \vDash_{\Gamma'} \gamma$ and $ y \vDash_{\Gamma''} \gamma$.

    \item $\Gamma\R\define \Gamma'\R (=\Gamma''\R)$. \footnote{\FNF{this is just $\Gamma\R = \Gamma'\R (= \Gamma''\R)$}}

    \item $\Gamma_\sigma \define \Gamma'_\sigma$, where $\sigma \in \{src,\, tgt,\, refl \}$.
\end{itemize}
Define $\lozenge$ as the terminal object $\mathbf{1}_{\cORGph(\cAsm)}$. \\

Consider the scaling structure and let $\Gamma \define \rho(\Gamma')$ :
  \begin{itemize}
    \item $\Gamma_\sigma = (\carr{\Gamma'_\sigma}, \vDash_{\Gamma_\sigma})$ with $a \vDash_{\Gamma_\sigma} \gamma$ iff there is $x \in \algA$, s.t $a = !_\rho x$ and $x \vDash_{\Gamma'_\sigma} \gamma$ for $\sigma \in \{O, R\}$
    
    \item again, scaling leaves unmodified $\Gamma_\sigma$ for $\sigma\in  \{src,\, tgt,\, refl \}$.
\end{itemize}

Let $RTm(\Gamma, S)$ be the collection of morphisms from the terminal family $\terminal{\cFamORG(\cAsm, \Gamma)}$ to $S$. Spelling this out and simplifying it, an element of $RTm(\Gamma, S)$ is a tuple of functions $(M\O: \forall \gamma_o \in \Gamma\O.\lfloor S\O(\gamma_o) \rfloor, M\R : \forall \gamma_r \in \Gamma\R. S\R(\gamma_r))$, s.t.\ the conditions from \cref{def:fam-refl-graphs} are satisfied, namely \footnote{\FNF{would be good for clarity to spell this out (up to isomorphism, removing the unit type element)}}:
\begin{align*}
\forall \gamma_o \in \Gamma\O. {S\rfl} (\widehat{M\O}(\gamma_o)) &= M\R(\Gamma\rfl(\gamma_o))\\
\forall \gamma_r \in \Gamma\R. {S\src} (M\R(\gamma_r)) &= \widehat{M\O}(\Gamma\src(\gamma_r))\\
\forall \gamma_r \in \Gamma\R. {S\tgt} (M\R(\gamma_r)) &= \widehat{M\O}(\Gamma\tgt(\gamma_r))
\end{align*}

$U_{\Gamma,S}$ is the just identity function.\\
Substitution in terms is given by precomposition with $f : \Gamma \to \Delta$, let $-\{f\O\} \define \lambda M\O. \forall \gamma \in \Gamma . M\O(f(\gamma))$ and similarly, $-\{f\R\} \define \lambda M\R . \forall \gamma \in \Gamma . M\R(f(\gamma))$. The functor $U$ interacts nicely with the so-defined $-\{f\}$ as essentially the substitution in terms in the underlying CwF is defined in the same way.\\

Resourced context extension is given by $\Gamma . \rho S \define (\Gamma'\O, \Gamma'\R, \Gamma'_\sigma)$, where
\begin{itemize}
  \item $|\Gamma'\O| \define \{(\gamma, x) : \gamma \in |\Gamma\O|, x \in S\O(\gamma) \}$\\
  $a \vDash_{\Gamma. \rho S} (\gamma, x)$ iff there exists $b, c \in \algA$, s.t $a = [b, !_\rho c]$, $b \vDash_{\Gamma} \gamma$ and $c \vDash_{S(\gamma)} \pi_1(\check{(\gamma, x)}) $, where $\check{(-)} : \Gamma . S \to U(\Gamma . S)$, $\check{(-)} \define id$ as the set-theoretic part of the extensions in the CwF and $\cL$ is the same by definition.
  
  \item $\Gamma'\R \define \{(\gamma, r) : \gamma \in \Gamma\R, r \in S\R(\gamma)\}$ \footnote{\FNF{no realizability anymore}}
  
  \item Each $\Gamma'_\sigma$ is defined pointwise.
\end{itemize}
            
The natural transformation $emp_\pi : \lozenge \to \pi \lozenge$ is given by the identity functions on both the object and relational part. It is realised by $K. !_\rho I$.

We list the realisers for the remaining transformations:
\begin{itemize}[noitemsep]
  \item $emp_+$ - $K . [I, I]$,
  \item $ext_\pi$ -  $\lambda^* q. \textit{let }[x, y] = q\textit{ in } F_\pi . (F_\pi. (!_\pi \lambda^* s t u. u s t) . x) . \delta_{\pi \rho} y$ 
  \item $ext_+$ - $\lambda^* q. \textit{let } [[x , y], z] = q \textit{ in } W_{\pi\rho}.(\lambda^* a b. [[x, a],  [y, b]]).z$
\end{itemize}
The underlying function part of the above-defined natural transformations is given by identity, hence naturality follows trivially.\\

Finally, we construct the morphisms, listed in \cref{def:qcwf-morphisms}:
\begin{itemize}
  \item $p_{\Gamma.S}: \Gamma. 0 S \to \Gamma$ is the first projection of $(\Gamma. 0 S)_\sigma = \{ (\gamma, s) : \gamma \in \Gamma_\sigma. s \in S(\gamma)\}$, ($\sigma \in \{\O, \R \}$) and is realized by $\lambda^* t. (t.K)$.\\
  The equality $U(p_{\Gamma.S}) = p_{U\Gamma.S}$ holds trivially due to the identical structure of context extension in the underlying $CwF$ and $\cL$.
  
  \item define $v_{\Gamma.S} \in RTm(0\Gamma.1S, S\{p_{U\Gamma.S}\})$ as the second projection. $v_{\Gamma.S}$ is realized by $\lambda^* t.\ B.t.K.D$.
  
  \item Let $a^\sigma_f$ realize $f_\sigma$, then $wk(f, \rho S') \define \lambda (\gamma, s). (f (\gamma), s)$ is realized by $\lambda^* q. \textit{let }[x, y] = q \textit{ in } [a^\sigma_f . x , y]$
  
  \item given a $M_\sigma \in RTm(\Gamma, S) = M_\sigma : \forall \gamma \in U(\Gamma_\sigma). S_\sigma(\gamma)$ with realizers $a^\sigma_m$, let $\overline{\rho M}_\sigma \define \lambda \gamma. (\gamma, M_\sigma(\gamma))$ realized by $\lambda^* q. \textit{let }[x, y] = q \textit{ in } [x, F_\rho.(!_\rho a^\sigma_m).y]$   
  
  \item given a $M_\sigma \in Tm(U\Gamma, S) = M_\sigma : \forall \gamma \in U\Gamma_\sigma. S_\sigma(\gamma)$, let $\overline{M}_\sigma \define \lambda \gamma. (\gamma, M_\sigma(\check\gamma))$ realized by the $K$ combinator.
\end{itemize}

From now on, we refer to the constructed model as $\dM$.

\subsection*{Type formers}

% \paragraph{Universe of small types}

\begin{definition}[Dependent function types a QCwF]
  \label{def:prod-types}
  A QCwF supports dependent function types with usage information, if
  \begin{itemize}
    
    \item the underlying CwF \cC supports them, namely, if for all $\Gamma \in \ob\cC, S \in Ty(\Gamma), T \in Ty(\Gamma . S)$, $\pi \in R$, there exist type $\Pi\pi S T \in Ty(\Gamma)$ and a bijection $$\Lambda : Tm(\Gamma.S, T) \cong Tm(\Gamma, \Pi\pi S T) \label{eq:dep-prod-cwf},$$ natural in $\Gamma$.
    
    \item for $\Gamma \in \ob\cL, S \in Ty(U\Gamma), T \in Ty(U\Gamma . S), \pi \in R$, there exists a bijection $$\Lambda_\cL : RTm(\Gamma.\pi S, T) \cong RTm(\Gamma, \Pi\pi S T) \label{eq:dep-prod-qcwf},$$ natural in $\Gamma$ such that $U \comp \Lambda_\cL = \Lambda \comp U$ and $U \comp \Lambda^{-1}_\cL = \Lambda^{-1}\comp U$.    
  
  \end{itemize}  
\end{definition}

To show that our model supports $\Pi$ types, fix some $\pi \in R$, suppose $\Gamma$ is a r.g in $\ob\cC$, $S \in Ty(\Gamma)$, $T \in Ty(\Gamma.S)$. 
Define the semantic type $\Pi \pi S T$ as the family of assemblies over $\Gamma$, consisting of:

\begin{itemize}
  \item $(\Pi \pi S T)\O(\gamma) \define (X, \vDash_X)$ for $\gamma \in \Gamma\O$, where
    \subitem
        $\begin{aligned}
          X \define \{ (f\O,& f\R) \quad | \quad \\ 
          & f\O : \forall s \in \lfloor S\O(\gamma) \rfloor.T\O(\gamma, s),\\
          & f\R : \forall r \in S\R(\Gamma\rfl(\gamma)). T\R(\Gamma\rfl(\gamma), r),\\
          & \exists \widehat{f\O} .\quad  (\forall s \in S\O(\gamma). T\rfl(\gamma, s)(\widehat{f\O} (s)) = f\R(S\rfl(\gamma)(s)) \\
          & \wedge\, \forall r \in S\R(\Gamma\rfl(\gamma)). T\src(\Gamma\rfl(\gamma),r)(f\R (r)) = \widehat{f\O}(S\src(\Gamma\rfl(\gamma))(r)) \\
          & \wedge\, \forall r \in S\R(\Gamma\rfl(\gamma)). T\tgt(\Gamma\rfl(\gamma),r)(f\R (r)) = \widehat{f\O}(S\tgt(\Gamma\rfl(\gamma))(r)))  \}
        \end{aligned}$\\
    
    \subitem $a \vDash_X (f\O, f\R)$ iff
    $$\forall s \in \lfloor{S\O(\gamma)}\rfloor, b \in \algA . b \vDash_{S\O(\gamma)} s \implies a . !_\rho b \vDash_{T\O(\gamma,s)} f\O(s)$$
    Note that $f\R$ does not contribute any realisability information to $\vDash_X$.
    
    \item
     \begin{align*}
           (\Pi \pi S T)\R(\gamma)  \define \{( (f\O^{src}, \,& f\R^{src}), (f\O^{tgt}, f\R^{tgt}), r)  \quad|\quad \\
            & (f\O^{src},f\R^{src}) \in (\Pi\pi S T)\O(\Gamma\src(\gamma)),\\
            & (f\O^{tgt},f\R^{tgt}) \in (\Pi\pi S T)\O(\Gamma\tgt(\gamma)),\\
            & r : \forall s \in S\R(\gamma). T\R(\gamma, s),\\     
            & \forall s \in S\R(\gamma). T\src(\gamma, s)( r (s)) = \widehat{f\O^{src}}(S\src(\gamma)(s)),\\
            & \forall s \in S\R(\gamma). T\tgt(\gamma, s)( r (s)) = \widehat{f\O^{tgt}}(S\tgt(\gamma)(s))\footnote{\GN{THAT DOES NOT ACTUALLY WORK!}} \}
      \end{align*}
%    
%    \subitem $a \vDash_Y (f^{src}, f^{tgt}, r)$ iff there exists $a\src, a\tgt, a\R$, s.t $a = [a\src, a\tgt, a\R]$ with $a\src \vDash_{(\Pi\pi S T)\O(\Gamma\src(\gamma))} f^{src}$, $a\tgt \vDash_{(\Pi\pi S T)\O(\Gamma\tgt(\gamma))} f^{tgt}$ and
%    $$\forall s \in \carr{S\R(\gamma)}, b \in \algA. b \vDash_{S\R(\gamma)} s \implies a\R . !_\rho b \vDash_{T\R(\gamma,s)} r(s)$$
%     

    \item $(\Pi \pi S T)\rfl(\gamma) \define \lambda (f\O, f\R). ((f\O, f\R), (f\O, f\R), f\R) $ for $\gamma \in \Gamma\O$.
    
    \item $(\Pi \pi S T)\src(\gamma) \define \lambda (f^{src}, f^{tgt}, r). f^{src}$ for $\gamma \in \Gamma\R$.
    
    \item $(\Pi \pi S T)\tgt(\gamma) \define \lambda (f^{src}, f^{tgt}, r), f^{tgt}$ for $\gamma \in \Gamma\R$.
\end{itemize}
%To ensure $(\Pi \pi S T)$ is a well-defined family of r.g of assemblies w.r.t to \cref{def:fam-refl-graphs}, let $\gamma \in \Gamma\O$
%%%%%%%%%%%%%%%%%%%%%%ee

Unwinding \cref{eq:dep-prod-qcwf}, we get 
$$\{(M\O,M\R) : \forall (\gamma, s) \in (\Gamma. \pi S) . T(\gamma, s) \} \cong \{(N\O, N\R) : \forall \gamma \in \Gamma . (\Pi \pi S T)(\gamma)\}$$
where the terms are of the following type structure:
\begin{align*}
  M\O &: \forall (\gamma, s) \in (\Gamma. \pi S)\O . T\O(\gamma, s) \\
  M\R &: \forall (\gamma, r) \in (\Gamma. \pi S)\R . T\R(\gamma, s) \\
  N\O &: \forall \gamma \in \Gamma\O .\\
  &\{ (f\O , f\R) \quad|\quad f\O : \Pi S(\gamma)\O.T(\gamma)\O\\
  &\hspace{6.2em}  f\R : \Pi S\R (\Gamma\rfl(\gamma).T(\Gamma\rfl(\gamma))) \} \\
  N\R &: \forall \gamma \in \Gamma\R.\\
  &\{ (f^{src}, f^{tgt} , r) \quad|\quad  f^{src} \in (\Pi \pi S T)\O(\Gamma\src(\gamma))\\
  &\hspace{8.1em}   f^{tgt} \in (\Pi \pi S T)\O(\Gamma\tgt(\gamma))\\
  &\hspace{8.2em}   r : \Pi S\R (\gamma).T\R(\gamma)\}
\end{align*}

Thus, we can define $\Lambda$ as $\Lambda(M\O,M\R) = (N\O, N\R)$, where

$$N\O \define \lambda \gamma_o . (\lambda s. M\O(\gamma_o, s), \lambda s_r. M\R(\Gamma\rfl(\gamma_o), s_r))$$
$$N\R \define \lambda \gamma_r . ( N\O(\Gamma\src(\gamma)), N\O(\Gamma\tgt(\gamma)),  \lambda s_r. M\R(\gamma, s_r)) )$$

For $\Lambda_\cL$, a realizer $a_m$ of $M$\\
(\textit{that is } $\mathit{\forall (\gamma,s), \forall (a_\gamma, a_s), [a_\gamma, a_s] \vDash_{\Gamma. \pi S} (\gamma,s) \implies a_m. [a_\gamma, a_s] \vDash_{T(\gamma,s)} M(\gamma,s)}$)\\
can be transformed to a realizer $a_n$ of $N$ by:
$$a_n \define \lambda^* y. (\lambda^* s . (a_m . [y, s]))$$ 
The conditions $U \comp \Lambda_\cL = \Lambda \comp U$ and $U \comp \Lambda^{-1}_\cL = \Lambda^{-1}\comp U$ follow trivially.

\paragraph*{Universe of small types}
A plausible candidate for the universe $U$ is the general definition in \cite{fill}:\\
$\begin{aligned}
  U\O &\define \text{ the set of small r.g. } \\
  U\R &\define \{(A, B, R, R\src, R\tgt) : R \in \cU, R\src: R \to A\O, R\tgt : R \to A\O, A,B \text{ are small r.g.}   \}
\end{aligned}$.\\

However, this universe turns out to be ``too big' - types do not carry enough structure to model parametricity accurately. To remedy the definition, we impose restrictions laid out by the following characterization:\\

A family of reflexive graphs of assemblies $S$ over $\Gamma$ is
\begin{itemize}
  \item small - if for all $\gamma_\sigma \in \Gamma_\sigma$, $S\R(\gamma\R) \in \cU$ and $\carr{S\O(\gamma\O)} \in \cU$.
  
  \item discrete - if for every $\gamma \in \Gamma\O$, there exists $X \in \cAsm$, s.t.
   
  % $$(S\O(\gamma), S\R(\Gamma\rfl(\gamma)), S\rfl(\gamma), S\src(\Gamma\rfl(\gamma)),  S\tgt(\Gamma\rfl(\gamma))) \cong (X, \carr{X}, id, id, id)$$
  % https://tikzcd.yichuanshen.de/#N4Igdg9gJgpgziAXAbVABwnAlgFyxMJZABgBpiBdUkANwEMAbAVxiRAGUAdTgeQApuAczoBbEXQCUIAL6l0mXPkIoyAJiq1GLNl04AlAZwDio8dwBOAMwaHhYyVNnzseAkQDM5DfWatEIAA0ZORAMFyUPUnVqH21-bgBjOnNzYADpYOdFNxRVUgBGby0-EESCQRkNGChBeCJQS3MIESR86hwIJDIQACMYMCgkABYAThjinW44cwTDE3sLa1tTBxBqBjo+hgAFBVdlEAYYSxxMkEbmrvbOxDbNX0nOKxshFal1zZgdvYj-LDBsLA1iA4AALLAnLpOc5NFq3a5XXr9QaIAC0o3GD38umms248zMTyWr3sEneh0+33CORA5iwglBp2oYIhp0QxGhFzheRAHSQngpW121IO-0BrExcRAWEGnNh-IRiB5fQG-O6sRKMrOXIVvJuyuRSFR7nVEz+gw+Qp+NLpDNO0go0iAA
  \begin{tikzcd}
  S\O(\gamma) \arrow[dd, "S\rfl(\gamma)" description]                                                                                   &  &       & X \arrow[dd, "id" description]                                     \\
  &  & \cong &                                                                    \\
  S\R(\Gamma\rfl(\gamma)) \arrow[uu, "S\src(\Gamma\rfl(\gamma))", bend left=49] \arrow[uu, "S\src(\Gamma\rfl(\gamma))"', bend right=49] &  &       & \carr{X} \arrow[uu, "id", bend left] \arrow[uu, "id"', bend right]
  \end{tikzcd}
  
  \item proof-irrelevant - if for all $\gamma \in \Gamma\R$, the function $\langle S\src(\gamma), S\tgt(\gamma) \rangle : S\R(\gamma) \to \carr{S\O(\Gamma\src(\gamma))} \times \carr{S\O(\Gamma\tgt(\gamma))}$ is injective.
\end{itemize}
For any reflexive graph $\Gamma$ , define the small, discrete, proof-irrelevant universe $U \in Ty(\Gamma)$ and the type decoder $T \in Ty(\Gamma . U)$ as:
\begin{itemize}
  \item $\carr{U\O(\gamma)} \define \text{the set of small, discrete r.g. of assemblies}$ \\
   $a \vDash_{U\O(\gamma)} S$ for $a \in \algA$, $a = I$ and $S \in \carr{U\O(\gamma)}$. \footnote{\FNF{Again maybe $a \vdash S$ iff $a = I$ is better for linearity reasons?}}
   
  \item $\begin{aligned}
  U\R(\gamma\R) \define \{(S,T, R, & R\src, R\tgt) \quad|\quad R \in \cU \\
  & S,T \text{ are small discrete r.g. of assemblies} \\
  & \langle R\src, R\tgt\rangle : R \to \carr{S\O} \times \carr{T\O} \textit { is injective} \}
  \end{aligned}$
  
  \item $U\rfl(\gamma\R)(S) \define (S, \carr{S}, S\rfl, S\src, S\tgt)$
  
  \item $U\src(\gamma\R)(S, T, R, R\src, R\tgt) = S$
  
  \item $U\src(\gamma\R)(S, T, R, R\src, R\tgt) = T$
\end{itemize}
and $T \in Ty(\Gamma . U)$ as:
\begin{itemize}
  \item $T\O(\gamma\O, S) \define S\O$
  \item $T\R(\gamma\R, (S,T, R, R\src, R\tgt)) \define R$
  \item $T\rfl(\gamma\O, S) \define S\rfl$
  \item $T\src(\gamma\R, (S,T, R, R\src, R\tgt)) \define R\src$
  \item $T\tgt(\gamma\R, (S,T, R, R\src, R\tgt)) \define R\tgt$.
\end{itemize}
\begin{claim} $U$ is closed under $\Pi$ types.
\end{claim}
Given some r.g $\Gamma$ and $S \in Ty(\Gamma)$, $T \in Ty(\Gamma.S)$, it suffices to show that $\Pi \pi T S \in Ty(\Gamma)$ is a small, discrete and proof-irrelevant family of r.g. of assemblies$^\dagger$ \footnote{it actually does not suffice, we implicitly assume a lemma that gives an isomorphism between terms of type $U$ and small, discrete, p.i. r.g of assemblies$^\dagger$} For brevity, let $V \define \Pi \pi S T$.\\

Smallness follows by the closure under $\Pi$-types in the ambient set-theoretical universe $\cU$.\\
For proof-irrelevance, take some $\gamma\R \in \Gamma$ and $f , g \in V\R(\gamma\R)$, s.t $\langle V\src(\gamma\R), V\tgt(\gamma\R) \rangle (f) = \langle V\src(\gamma\R), V\tgt(\gamma\R) \rangle (g)$. WTP $f = g$, by def. we get immediately that $f^{src} = g^{src}$ and $f^{tgt} = g^{tgt}$. 

Given that $(f^{src}, f^{tgt}, r_f), (f^{src}, f^{tgt}, r_g) \in V\R(\gamma\R)$, note that , 
\begin{align*}
  \forall s \in S\R(\gamma\R). \langle T\src(\gamma\R, s), T\src(\gamma\R, s) \rangle  ( r_f(s)) = (f\O^{src}(S\src(\gamma\R)(s)), f\O^{tgt}(S\tgt(\gamma\R)(s)))\\
  \forall s \in S\R(\gamma\R). \langle T\src(\gamma\R, s), T\src(\gamma\R, s) \rangle  ( r_g(s)) = (f\O^{src}(S\src(\gamma\R)(s)), f\O^{tgt}(S\tgt(\gamma\R)(s)))
\end{align*}

Since $T$ is proof-irrelevant, it follows directly that $r_f = r_g$ and thus $f = g$. 

\paragraph{A free theorem}
%\begin{thm}
%  If $\Gamma \vdash M : \Pi a \overset{0}{:} \bU . \Pi \_ \overset{0}{:} \bT a . \bT a$, 
%  then for any $\Gamma \vdash X : \bU$, $\Gamma \vdash M X  = \mathbf{0} : \bT X \to \bT X$ is sound when interpreted in the model constructed so far.
%\end{thm}
%\textbf{Sketch } Instantiate $M\R$ with $\Gamma\rfl(\gamma\O)$ and uncurry once to get:\\
%$M\R(\Gamma\rfl(\gamma\O), -): \forall R \in \bU\R(\Gamma\rfl(\gamma\O)). \Pi 0 (T\R(\Gamma\rfl(\gamma\O)), R) . (T\R(\Gamma\rfl(\gamma\O)), R)$ (1)\\
%By the def. of terms in our model, we have:\\
%$U\src(\Gamma\rfl(\gamma\O)). M\R(\Gamma\rfl(\gamma\O)) = M\O(\gamma\O)$\\
%$U\tgt(\Gamma\rfl(\gamma\O)). M\R(\Gamma\rfl(\gamma\O)) = M\O(\gamma\O)$\\
%Expanding the def. in (1), we get:\\
%$$\forall A, B \in U\O(\gamma\O), \forall R \subseteq A\O \times B\O,$$
%$$(M\O(\gamma\O)(A\O),  M\O(\gamma\O)(B\O) ) \in (\Pi 0 R . R)\R$$
%Instantiate $B$ with $\mathbf{0}$.\\
%

\begin{definition}[No universal realizer]
  Given a context $\Gamma$ and a type $T$, s.t $\Gamma \vdash  T$, the model $\cC$ constructed so far has no universal realizer iff $\bigcap\limits_{\gamma \in \llbracket\Gamma\rrbracket\O} \{ a \in \algA :  \text{for every x} \in \carr{\llbracket T\rrbracket \O(\gamma)}, a \vDash_{\llbracket T \rrbracket \O(\gamma\O)} x\}$ is empty.
\end{definition}

\begin{thm}
  Let $\Gamma$ be a context and $T \define \Pi a \overset{0}{:} \bU . \Pi \_ \overset{0}{:} \bT a . \bT a$  - a type. Assume the model has no universal realizers. There is no resourced term $M$ of that type - i.e. $\Gamma \vdash M \overset{1}{:} T$ does not hold in $\dM$.
\end{thm}

Assume such term $M$, $\Gamma \vdash M \overset{1}{:} T$ exists. Fix some $\gamma \in \Gamma\O$ and consider the uncurried term $M'$, s.t $\Gamma, a \overset{O}{:} \bU, \_ \overset{0}: \bT a \vdash M' \overset{1}{:} \bT a$ and
$$ M'\O(\gamma\O, a\O, \_) \define \textit{ let } ((f''\O, f''\R) , f'\R)= M\O(\gamma\O) \textit{ in } f''\O(\gamma\O, a\O, \_)$$
$$ M'\R(\gamma\R, a\R, \_) \define \textit{ let } (f'^{\src}, f'^{\tgt}, (f''^{\src}, f''^{\tgt}, r)) = M\R(\gamma\R) \textit{ in } r(\gamma\R, a\R, \_)$$

Fix some $\gamma\O \in \Gamma\O$. Spelling out explicitly the type of $M'\R$, ``instantiated'' at $\Gamma\rfl(\gamma\O)$ (or equivalently, of $r(\Gamma\rfl(\gamma\O), - , -)$ and suppressing the realizability information, we get that:
$$M'\R(\Gamma\rfl(\gamma\O), -, -) : \forall a\R \in \bU\R(\Gamma\rfl(\gamma\O)). \bT\R(\Gamma\rfl(\gamma\O), a\R) \to \bT\R(\Gamma\rfl(\gamma\O), a\R)$$
Unpacking the definition of $\bU\R(\Gamma\rfl(\gamma\O))$, we get (by conditions in \cref{def:fam-refl-graphs}):
$$\forall S, T \in \bU\O(\gamma\O), \forall R \in \cU , R\src : R \to S\O, R\tgt : R \to T\O:$$
$$R\src(M'\R(\Gamma\rfl(\gamma\O),(S,T,R, R\src, R\tgt), (s,t))) = M'\O(\gamma\O, S, s)$$
$$R\tgt(M'\R(\Gamma\rfl(\gamma\O),(S,T,R, R\src, R\tgt),  (s,t))) = M'\O(\gamma\O, T, t)$$
Thus, we conclude that 
\begin{align}
\forall S, T \in \bU\O(\gamma\O), \forall R \subseteq S \times T,  \forall (s,t) \in R.\ (M'\O(\gamma\O, S, s), M'\O(\gamma\O, T, t)) \in R
\end{align}

Let $X$ be some type, s.t $\Gamma \vdash X : \bU$. Consider the term $M$ instantiated at $\gamma\O$ and $X$ and $R^X \define \{(x,x) | x : X\}$. Substituting $X$ for $S$ and $T$, $R^X$ for $R$ in $(1)$ and applying currying, we get that for each $x : X$, $(M(\gamma\O, X(\gamma\O), x), M(\gamma\O, X(\gamma\O), x)) \in R$ holds. Hence $M(\gamma\O, X(\gamma\O), x) = x$.\\

Now since $M \in RTm(\Gamma, T)$, $M$ is realizable - in particular, there exists an $a \in \algA$ that tracks $M(\gamma\O X(\gamma\O), -)$. By def.\ we get that $\forall x \in X\O, \forall b \in \algA. b \vDash_{\Gamma.0 X} (\gamma\O, x) \implies a . !_0 b \vDash_{X(\gamma\O)} M(\gamma\O, X(\gamma\O), x)$ - $a$ is a realizer for every element $x$ in $X\O(\gamma\O)$. But that is a contradiction, as no universal realizer exists for $X$ by assumption. Therefore no resourced term $M \overset{1}{:} T$ exists.

\begin{definition}[Separability of realizers]
  Let $\Gamma$ be a context and $S \in Ty(\Gamma)$. $S$ has separable realizers if for every $\gamma \in \Gamma\O$ and $x, y  \in |S\O(\gamma)|$ and $a \in \algA$, if $a \vDash_{S\O(\gamma)} x$ and $a \vDash_{S\O(\gamma)} y$, then $x = y$.
\end{definition}

\begin{thm}
    Let $\Gamma \vdash A : \bU$ and $\Gamma, a \overset{\sigma}{:} A \vdash B : \bU$. If $A$ has a realisable inhabitant and $B$ has separable realisers, then $ \llbracket B \rrbracket \cong \llbracket \Pi x \overset{0}:A.\, B \rrbracket$ holds in $\dM$.
\end{thm}
Let $a^*$ be the realisable inhabitant of $A$, $a^* \in RTm(\Gamma, A)$, $a^*_o : \forall \gamma_o \in \Gamma\O.\lfloor A\O(\gamma_o) \rfloor$,\\ $a^*_r : \forall \gamma_r \in \Gamma\R . A\R(\gamma_r)$. \\
Define the morphisms $g : \Pi x \overset{0}:A.\ B \to B$ and $h : B \to \Pi x \overset{0}:A.\ B$ as \footnote{for the time being, focus only on the object part}:
\begin{align*}
{g\O}(\gamma_o) &\define \lambda (f\O, f\R). f\O(a^*_o(\gamma)) && {h\O}(\gamma_o) \define \lambda b\O. (\lambda a_o. b\O, \lambda a_r. B\rfl(b\O))
\end{align*}

Let $V \define \Pi x \overset{0}:A.\ B$, observe that if $(f\O, f'\R), (f\O, f''\R) \in |V\O|$, then $f'\R = f''\R$ using the proof-irrelevance of $B$. Hence, to show that two inhabitants of $|V\O|$ are equal, it suffices to prove it for the first components only (1).\\

Fix some $(f\O, f\R) \in \lfloor V\O \rfloor$. To find a realiser of $g\O(\gamma_o)$, consider the realiser $a_f \vDash_{V\O(\gamma_o)} (f\O, f\R)$. By def. we have that $\forall x \in |A\O(\gamma_O)|, a \in \algA . a \vDash_{A\O(\gamma_o)} x \implies a_f . !_0 a \vDash_{B\O(\gamma_o)} f\O(x)$. As $B$ has separable realisers, it must be the case that $f\O$ is a constant function (2). Define the realiser $a_g$ of $g\O(\gamma_o)$ as $a_g \define \lambda^* x. (x . I)$.

As for $h\O(\gamma_o)$, we must first ensure that $h\O$ outputs well-defined inhabitants of $V\O(\gamma_o)$. Notice that both $f' \define \lambda a_o. b\O$ and $f'' \define \lambda a_r. B\rfl(b\O)$ are well-typed and we can easily verify they satisfy the conditions in \cref{def:prod-types} by direct substitution (where the extension of $f'$ is the natural extension over $|A\O|$). To construct the realiser of $f'$, let $a_b \vDash_{B\O(\gamma_o)} b\O$ (such $a_b$ exists due to $b\O \in \lfloor B\O(\gamma_o)\rfloor)$, then $K. a_b$ realises $f'$.\\

Let $b^* \in \lfloor B\O(\gamma_o) \rfloor$, by def. we have that $ (g\O \comp h\O)(\gamma_o, b^*) = b^*$. As for the converse direction, let $(h\O \comp  g\O)(\gamma_o)(f\O, f\R)=(f'\O, f'\R)$ for some $(f\O, f\R) \in V\O(\gamma_o)$. By (1), we show only that $f\O = f'\O$. As $f\O$ is a constant function by (2), expand the definition of $h_o$ to obtain immediately $f\o = f'\O$.

%\footnote{In Tarski's fashion we make distinction between types and terms by introducing a type U for codes of types and explicit decoder T}
\printbibliography

\end{document}
