\documentclass[12pt,a4paper]{article}
\usepackage{amssymb}
\usepackage{amsmath}
\usepackage{amsthm}
\usepackage{xspace}
\usepackage{tikz}
\usepackage{tikz-cd}
\usepackage{enumitem}
\usepackage{adjustbox}
\usepackage{stmaryrd}
\usepackage{mathrsfs}  
\usepackage{proof}
\usepackage[left=5em, right=5em]{geometry}

%%%%%%%%%%%%%%%%%%%%%%%%%%%%%%%%%%%% Link colors %%%%%%%%%%%%%%%%%%%%%%%%%%%%%%%%%%%
\usepackage{hyperref}
\usepackage[capitalize,nameinlink]{cleveref}

\hypersetup{
  colorlinks,
  citecolor=black,
  filecolor=black,
  linkcolor=black,
  urlcolor=black,
  linktoc=all
}

%%%%%%%%%%%%%%%%%%%%%%%%%%%%%%%%%%% Bibliography %%%%%%%%%%%%%%%%%%%%%%%%%%%%%%%%%%%

\usepackage[backend=biber, isbn=false, doi=false, style=numeric, 
citestyle=numeric, giveninits=true]{biblatex}
\bibliography{lib.bib}

%%%%%%%%%%%%%%%%%%%%%%%%%%%%%%%% Theorem enivornment %%%%%%%%%%%%%%%%%%%%%%%%%%%%%%%

%use continious uniform numbering

\newtheorem{thm}{Theorem}[section]
\newtheorem{prop}[thm]{Proposition}
\newtheorem{cor}[thm]{Corollary}
\newtheorem{lemma}[thm]{Lemma}
\newtheorem*{claim}{Claim}
\newtheorem*{fact}{Fact}

\theoremstyle{definition}
\newtheorem{remark}[thm]{Remark}
\newtheorem{definition}[thm]{Definition}
%% remove section prefix
\renewcommand{\thethm}{\arabic{thm}}
% \makeatletter
%\def\th@definition{%
%  \thm@notefont{\normalfont\itshape}
%}
%\makeatother


%%%%%%%%%%%%%%%%%%%%%%%%%%%%%%%%%%%% Always math %%%%%%%%%%%%%%%%%%%%%%%%%%%%%%%%%%%

\makeatletter
\let\ea\expandafter

\def\mdef#1#2{\ea\ea\ea\gdef\ea\ea\noexpand#1\ea{\ea\ensuremath\ea{#2}\xspace}}
\def\alwaysmath#1{\ea\ea\ea\global\ea\ea\ea\let\ea\ea\csname your@#1\endcsname\csname #1\endcsname
  \ea\def\csname #1\endcsname{\ensuremath{\csname your@#1\endcsname}\xspace}}

\makeatother

%%%%%%%%%%%%%%%%%%%%%%%%%%%%%%%% Letter decorations %%%%%%%%%%%%%%%%%%%%%%%%%%%%%%%%

\makeatletter

\newcount\foreachcount

\def\foreachletter#1#2#3{\foreachcount=#1
  \ea\loop\ea\ea\ea#3\@alph\foreachcount
  \advance\foreachcount by 1
  \ifnum\foreachcount<#2\repeat}

\def\foreachLetter#1#2#3{\foreachcount=#1
  \ea\loop\ea\ea\ea#3\@Alph\foreachcount
  \advance\foreachcount by 1
  \ifnum\foreachcount<#2\repeat}

% Particular commands for typefaces, sometimes with the first letter
% different.
\def\autofmt@n#1\autofmt@end{\mathrm{#1}}
\def\autofmt@b#1\autofmt@end{\mathbf{#1}}
\def\autofmt@d#1#2\autofmt@end{\mathbb{#1}\mathsf{#2}}
\def\autofmt@c#1#2\autofmt@end{\mathcal{#1}\mathit{#2}}
\def\autofmt@s#1#2\autofmt@end{\mathscr{#1}\mathit{#2}}
\def\autofmt@f#1\autofmt@end{\mathsf{#1}}
%\def\autofmt@k#1\autofmt@end{\mathfrak{#1}}
%% Particular commands for decorations.
\def\autofmt@u#1\autofmt@end{\underline{\smash{\mathsf{#1}}}}
\def\autofmt@U#1\autofmt@end{\underline{\underline{\smash{\mathsf{#1}}}}}
\def\autofmt@h#1\autofmt@end{\widehat{#1}}
\def\autofmt@r#1\autofmt@end{\overline{#1}}
\def\autofmt@t#1\autofmt@end{\widetilde{#1}}
\def\autofmt@k#1\autofmt@end{\check{#1}}

% Defining multi-letter commands.  Use this like so:
% \autodefs{\bSet\cCat\cCAT\kBicat\lProf}
\def\auto@drop#1{}
\def\autodef#1{\ea\ea\ea\@autodef\ea\ea\ea#1\ea\auto@drop\string#1\autodef@end}
\def\@autodef#1#2#3\autodef@end{%
  \ea\def\ea#1\ea{\ea\ensuremath\ea{\csname autofmt@#2\endcsname#3\autofmt@end}\xspace}}
\def\autodefs@end{blarg!}
\def\autodefs#1{\@autodefs#1\autodefs@end}
\def\@autodefs#1{\ifx#1\autodefs@end%
  \def\autodefs@next{}%
  \else%
  \def\autodefs@next{\autodef#1\@autodefs}%
  \fi\autodefs@next}

\makeatother

%%%%%%%%%%%%%%%%%%%%%%%%%%%%%%%%%%% Miscellaneous %%%%%%%%%%%%%%%%%%%%%%%%%%%%%%%%%%

\setlength{\parindent}{0pt}

\def\define{:=}

\let\comp\circ

\def\pow{\mathcal{P}}
\def\nat{\mathbb{N}}
\newcommand{\carr}[1]{\ensuremath{|#1|}}
\newcommand{\ob}[1]{\ensuremath{Ob(#1)}}

\renewcommand{\phi}{\varphi}
\alwaysmath{Gamma}


\newcommand{\terminal}[1]{\mathbf{1_{#1}}}
\newcommand{\of}[1]{\overset{#1}{:}}
\newcommand{\ofto}[1]{\overset{#1}{\to}}

\newcommand{\emptyCtxt}{\diamond}
\newcommand{\letin}[3]{\text{ let } (#1) = #2 \text{ in } #3 }

\def\definecat#1{\ea\gdef\csname c#1\endcsname{\ensuremath{\mathcal{#1}}\xspace}}
\foreachLetter{1}{27}{\definecat}

\def\definealg#1{\ea\gdef\csname alg#1\endcsname{\ensuremath{\mathscr{#1}}\xspace}}
\foreachLetter{1}{27}{\definealg}

%%%%%%%%%%%%%%%%%%%%%%%%%%%%%%%%%%% Comments %%%%%%%%%%%%%%%%%%%%%%%%%%%%%%%%%%%%%%%

\newcommand{\FNF}[1]{\textcolor{blue}{\textbf{FNF:} #1}}
\newcommand{\GN}[1]{\textcolor{magenta}{\textbf{GN:} #1}}
\newcommand{\comment}[1]{\textcolor{gray}{\textit{#1}}}


\autodefs{\bR\dN\binl\binr\bcase}
\title{\vspace{-5em}Outline}
\date{\vspace{-3em}}

\begin{document}
\maketitle

%Given a resource semiring $\bR$ with elements $s , t , u , \dots \in R$, define the following equivalence relation on the set $R^T \define \{0_\bR\} \cup \mu X. (R\setminus\{0_\bR\})  + X \times X$:
%\begin{gather*}
%  \tau \sim \rho \iff 
%  \begin{cases}
%     \tau = s,\ \rho = t \text{ and } s =_R t \\
%     \tau = s,\ \rho = (\rho, \rho'') \text{ and } \tau \sim \rho' \text{ and } \tau \sim \rho''\\
%     \tau = (\tau', \tau''),\ \rho = t \text{ and } \tau' \sim \rho \text { and } \tau'' \sim \rho \\
%     \tau = (\tau', \tau''),\ \rho = (\rho', \rho'') \text{ and } \tau' \sim \rho' \text { and } \tau'' \sim \rho''      
%  \end{cases}
%\end{gather*}
%\begin{claim}
%  $R^T/\sim$ can be endowed with usage semiring structure.
%\end{claim}
%Adjust the $R-LCA$ to account for the tree structure by augmenting the operation $!_\tau : \algA \to \algA$ and the elements $W_{\pi\rho}$, $F_\rho$, $\delta_{\pi\rho}$:
%\begin{gather*}
%  !^T_\tau x \define 
%  \begin{cases}
%   [\ulcorner true \urcorner, E(!_s x, !_s x)]  &\text{ if } s = contract(\tau) \\
%   [\ulcorner false \urcorner , E(!^T_{\tau'} x , !^T_{\tau''} x)],  &\text{ if } \langle\tau', \tau''\rangle = contract(\tau)\label{1}
%  \end{cases}
%\end{gather*}
%
%
%The outermost pairing with Bool allows us to inspect/ the shape of the resource annotation internally within the BCI algebra.
%
%Futhermore, it is also possible to define $W_{\pi\rho}$, $F_\rho$, $\delta_{\pi\rho}$, but that involves quite a few subtleties.\\
%
%As an example usage of tree resource annotation, let's define the semantics of $\oplus$-types and check the rules are sound wrt. to it.\\
%$\llbracket (A \oplus B)(\gamma)\rrbracket \define (|A(y)| \sqcup |B(y)|, \vDash_{A\oplus B(\gamma)} )$ , where \\
%
%$ a \vDash_{A \oplus B(\gamma)} (i , x)$ iff 
%$\begin{aligned}
%  &(\exists\, a_x \in \algA. a = [\ulcorner true \urcorner, a_x] \land a_x \vDash_{A(\gamma)} x \land \ulcorner true \urcorner \vDash_{Bool(\gamma)} i)\ \lor \\
%  & (\exists\, b_x \in \algA. a = [\ulcorner false \urcorner, b_x] \land b_x \vDash_{B(\gamma)} x \land \ulcorner false \urcorner \vDash_{Bool(\gamma)} i) 
%\end{aligned}$

%\begin{figure}[!h]
%\begin{align*}
%  \infer[\oplus\text{-type}]{0\Gamma \vdash A \oplus B}{
%    0\Gamma \vdash A &
%    0\Gamma \vdash B} & & 
%  \infer[\text{inl}]{\Gamma \vdash \mathbf{inl}\, M \of \sigma A \oplus B}{
%      \Gamma \vdash M \of{\sigma} A} & &
%    \infer[\text{inr}]{\Gamma \vdash \mathbf{inr}\, N \of \sigma A \oplus B}{
%      \Gamma \vdash N \of{\sigma} B
%    }
%  \end{align*}
%  \begin{gather*}
%  0\Gamma, x \of{0} A \oplus B \vdash C \\
%  \infer[\oplus \text{-elim}]{\Gamma' + \langle\rho, \pi\rangle \Gamma \vdash \mathbf{case}(M, T_a, T_b) \of{\sigma} C[M / x]}{
%    \Gamma \vdash M \of \sigma A \oplus B &&
%    \Gamma', a \of {\rho} A \vdash T_a \of{\sigma} C[\mathbf{inl}\,a/x] && 
%    \Gamma', b \of{\pi} B \vdash T_b \of{\sigma} C[\mathbf{inr}\, b/x] &&
%    0\Gamma = 0\Gamma'
%  }
%  \end{gather*}
%  \begin{gather*}
%    0\Gamma, x \of{0} A \oplus B \vdash C \\
%    \infer[\oplus \text{-comp}]{\Gamma' + \langle\rho, \pi\rangle (a' \of{1} A)\vdash  \mathbf{case}(\mathbf{inl}(a'), T_a, T_b) = T_a[a'/a]}{
%      a' \of{\tau} A,  a \of {\rho} A \vdash T_a \of{\sigma} C[\mathbf{inl}\,a/x] && 
%      \Gamma', b \of{\pi} B \vdash T_b \of{\sigma} C[\mathbf{inr}\, b/x] &&
%      0\Gamma = 0\Gamma'
%    }
%  \end{gather*}
%\caption{Rules for $\oplus$-types}
%\end{figure}
%\newpage

\section{Quantitative polynomial functors}
\subsection{Category of closed types and linear functions}
\label{sec:cat-closed-types}
Let $\cC$ be the category of closed types and linear functions $f : (x \of {1} X) \to Y $ for which derivations in QTT exist. Composition of morphisms $\Gamma \vdash f \of{\sigma} (x \of{1} X) \to Y$ and $\Gamma \vdash g \of{\sigma} (y \of{1} Y) \to Z$ is given by ordinary function composition $\Gamma \vdash \lambda x \of{1} X. g(f(x)) \of{\sigma} (x \of{1} X) \to Z $.

The linearity restriction on the morphisms does not lead to loss of expressiveness - functions with arbitrary resource annotations can be represented as linear ones via:

\subsection{Category of closed types and linear functions in a nonempty context}

Let $\Delta$ be an ``underlying context'', i.e.\ a context of the form $\Delta = 0\Gamma_0$ for some $\Gamma_0$. There is a category $\cC_\Delta$ where the objects are types $X$ such that $\Delta \vdash X \text{ type}$, and a morphism $X$ to $Y$ consists of pair $(\Gamma, f)$, where $\Gamma$ is a context such that $0\Gamma = \Delta$, and $\Gamma \vdash f : X \ofto{1} Y$.

\begin{itemize}
\item The identity morphism is given by $(\Delta, \lambda x . x)$;
\item Composition of $(\Gamma_2, g)$ and $(\Gamma_1, f)$ is given by $(\Gamma_1 + \Gamma_2, \lambda x . f(g(x)))$.
\end{itemize}

This is a category since $0\Delta = \Delta$, and context addition is associative, and with $\Gamma + \Delta = \Delta + \Gamma = \Gamma$. Note that the category of $\cC$ closed types from \cref{sec:cat-closed-types} is a special case $\cC = \cC_{\emptyCtxt}$ where $\Delta = \emptyCtxt$, because the only $\Gamma$ with $0\Gamma = \emptyCtxt$ is $\Gamma = \emptyCtxt$.

\begin{lemma}
  Fix $\Delta$ as above, and let $\Delta \vdash A \text{ type}$ and $\Delta, x \of{0} A \vdash B[x] \text{ type}$. The operation $F(X) = (a \of{1} A) \otimes (Ba \overset{1}\to X)$ is a functor $\cC_\Delta \to \cC_\Delta$.
\end{lemma}
\begin{proof}
  If $\Delta \vdash X \text{ type}$ then $\Delta \vdash F(X) \text{ type}$. On morphisms, we can define
  \[
    F(\Gamma, f) = (\Gamma, \lambda z . \letin{a, h}{z}{(a, \lambda b . f(h(b)))})
  \]
  as the following derivation shows:
  \[
    \infer{\Gamma \vdash \lambda z . \letin{a, h}{z}{(a, \lambda b . f(h(b)))} : F(X) \ofto{1} F(Y)}
       {\infer{\Gamma, z \of{1} F(X) \vdash \letin{a, h}{z}{(a, \lambda b . f(h(b)))} : F(Y)}
          { \infer{\Gamma, z \of{0} F(X), a \of{1} A, h \of{1} B[a] \ofto{1} X \vdash (a, \lambda b . f(h(b))) : F(Y)}{\infer{0\Gamma, z \of{0} F(X), a \of{1} A, h \of{0} B[a] \ofto{1} X \vdash a : A}{} & \quad \deduce{\cD}{\vdots}} & \quad \infer{0\Gamma, z \of{1} F(X) \vdash z : F(X)}{}}}
      \]
      where $\cD$ is the derivation
      \[
                 {
            \infer{\Gamma, a \of{0} A, h \of{1} B[a] \ofto{1} X \vdash \lambda b . f(h(b)) : B[a] \ofto{1} Y}{ \infer{\Gamma, a \of{0} A, h \of{1} B[a] \ofto{1} X, b \of{1} B[a]  \vdash f(h(b)) : Y}{ {\Gamma, a \of{0} A, h \of{0} B[a] \ofto{1} X, b \of{0} B[a] \vdash f : X \ofto{1} Y} & \quad \deduce{\cD'}{\vdots} }} }
        \]
        weakened by $z \of{0} F(X)$, where again $\cD'$ is the derivation
        \[
\infer{0\Gamma, a \of{0} A, h \of{1} B[a] \ofto{1} X, b \of{1} B[a] \vdash h(b) : X}{\infer{0\Gamma, a \of{0} A, h \of{1} B[a] \ofto{1} X, b \of{0} B[a] \vdash h : B[a] \ofto{1} X}{} & \quad \infer{0\Gamma, a \of{0} A, h \of{1} B[a] \ofto{0} X, b \of{1} B[a] \vdash b : B[a]}{}}
\]
similarly weakened. This is functorial by the $\eta$-rules for functions and pairs.
\end{proof}


\subsection{Internal representation}
Let $F$ be the polynomial functor mapping a type $X$ to $(a \of{1} A) \otimes (Ba \overset{1}\to X)$ and a function $f : X \overset{1}\to Y$ to $Ff : (a \of{1} A)\otimes(Ba \overset{1}{\to} X) \to (a \of{1} A)\otimes(Ba \overset{1}{\to } Y)$

If $D_f$ is the derivation of $f$, we show how to derive $Ff$. Given some $a : A$ and $h : B a \overset{1}\to X$, start by composing with $f$ and constructing a tensor product:

$$ 
\infer{\Gamma, a \of{1} A, h \of{1} B a \overset{1}\to X \vdash (a, \lambda b \of{1} B a. f(g(b))) \of{1} (a \of{1} A)\otimes(B a \overset{1}\to Y)}{
  \infer*{0\Gamma, a \of{1}, h \of{0} \dots \vdash a \of{1} A}{}
  & 
\infer*{\Gamma, a \of{0} A, h \of{1} B a \overset{1}\to X \vdash \lambda b \of{1} B a. f(g(b)) \of{1} (B a \overset{1}\to Y) }{
  \infer[Var]{0\Gamma, a \of{0} A, h \of{1} B a \overset{1}\to X \vdash h \of{1} B a \overset{1} \to X}{
      \vdash \Gamma, a \of{1} A, h \of{1} B a \overset{1}\to X
    }
  & 
  \infer{\Gamma \vdash f \of{1} X \overset{1}\to Y}{
    D_f
  }
}
}
$$


$$\infer[Lam]{\Gamma \vdash \lambda z \of {1} (a \of{1} A) \otimes (B a \to X). \text{ let } \dots \of{1} (a \of{1} A) \otimes (B a \overset{1} \to X) \overset{1}\longrightarrow (a \of{1} A) \otimes (B a \overset{1} \to Y)} {
  \infer[\oplus-E]{\Gamma,  z \of{1}  (a \of{1} A) \otimes(B a \overset{1}\to X) \vdash \text{let } (x,u) = a \text{ in } (x , \lambda b \of{1} Ba. f(u(b))) \of{1} (a \of{1} A)\otimes(B a \overset{1}\to Y)   }{
    \infer{0\Gamma, z \of{1} (a \of{1} A) \otimes (B a \overset{1}\to X) \vdash  z \of{1} \dots} {
      \vdash \Gamma, z \of{1} (a \of{1} A) \otimes (B a \overset{1}\to X)
    }
    &
    \Gamma, \dots \vdash (a, \lambda b \of{1} B a. f(g(b))) \of{1} (a \of{1} A)\otimes(B a \overset{1}\to Y)
  }
}
$$

\subsection{External representation (using adjoints)}
$(s \of{0} S) \otimes ((t \of{0} T) \otimes Id_{f(s) ,g(t) })) $
Use QCwF structure... 
\subsection{Generalising to non-empty contexts}
\subsection{Properties of quantitative polynomial functors}

\section{Algebras for QPFs}

\subsection{$\dN$}
$f : 1 \to Bool$, Let $A \define Bool$ and $B = \mathbf{1}$, $P_f$. 

$P_f(X) \define (a \of{1} Bool)\otimes((b \in f^{-1}(a)) \to X)$
Assume $N$, prove initiality.

If there is an initial algebra for $P_f$ there is a type $N$ that satisfies n.n. induction (requires dependent function).
\subsection{Lists}
\subsection{Trees}

\subsection{Induction principle}

\section{Rules for W-types in QTT}
\section{Parametricity and W-types}

\newpage
\section{Appendix}
\subsection{(Stand-alone) Sum types}
\begin{figure}[h]
  \begin{gather*}
    \begin{align*}
      \infer[\oplus\text{-type}]{0\Gamma \vdash \rho A \oplus \pi B}{
        0\Gamma \vdash A &
        0\Gamma \vdash B} & & 
      \infer[\text{inl}]{\rho \Gamma \ \vdash \mathbf{inl}\, S_1 \of \sigma \rho A \oplus \pi B}{
        \Gamma \vdash S_1 \of{\sigma} A} & &
      \infer[\text{inr}]{\pi\Gamma \vdash \mathbf{inr}\, S_2 \of \sigma \rho A \oplus \pi B}{
        \Gamma \vdash S_2 \of{\sigma} B
      }
    \end{align*}\\
    \\
    0\Gamma, x \of{0} \rho A \oplus \pi B \vdash C \\
    \infer[\oplus \text{-elim}]{\Gamma' + \Gamma \vdash \bcase(M , T_1 , T_2)\of{ \sigma}C[M/x]}{
      \Gamma \vdash M \of{\sigma} \rho A \oplus \pi B &&
      \Gamma', a \of {\rho} A \vdash T_1 \of{\sigma} C[\mathbf{inl}\, a/x] && 
      \Gamma', b \of{\pi} B \vdash T_2 \of{\sigma} C[\mathbf{inr}\, b/x] &&
      0\Gamma = 0\Gamma'    
    }\\
    \\
    \infer[\oplus \text{-comp}]{\Gamma' + \rho\Gamma \vdash \bcase(\binl(S_1), T_1, T_2) \equiv T_1[S_1 / a]}{
      \Gamma \vdash S_1 \of{\sigma} A &&
      \Gamma \vdash M \of{\sigma} \rho A \oplus \pi B &&
      \Gamma', a \of {\rho} A \vdash T_1 \of{\sigma} C[\mathbf{inl}\, a/x] && 
      0\Gamma = 0\Gamma'    
    }\\
  \end{gather*}
  \caption{Rules for $\oplus$-type}
\end{figure}
We give the following semantics for the $\oplus$-type:\\
$|\rho A \oplus \pi B\;(\gamma)| \define |A(\gamma)| \sqcup |B(\gamma)|$\\
$\begin{aligned}
  a \vDash_{\rho A \oplus \pi B\;(\gamma)} (i , x) \text{ iff } &(\exists b .  a =[!_\rho b, \ulcorner true \urcorner] \land b \vDash_{A(\gamma)} x \land i = 0)\; \lor \\
  &(\exists c .  a =[!_\pi c, \ulcorner false \urcorner] \land c \vDash_{B(\gamma)} x \land i = 1)
\end{aligned}$
\begin{claim}
  The rules are sound when interpreted wrt to the given semantics for $\oplus$-types and realisability model.
\end{claim}
\begin{proof}
  The underlying set-theoretic functions are immediate. For the realisers, let
  \begin{itemize}[noitemsep]
    \item $a_\binl \define \lambda^* x . [ F_\rho \cdot\ !_\rho a_{s_1}\cdot x , \ulcorner true \urcorner]$\\
    
    \emph{if $a_{s_1}\cdot {a_\gamma} \vDash  s_1$, then $a_\binl \cdot !_\rho a_\gamma \vDash \binl s_1$; $a_\binl \cdot !_\rho a_\gamma \leadsto [!_\rho (a_{s_1} \cdot a_\gamma) , \ulcorner true \urcorner]$}\\
    
    \item $a_\bcase \define 
    \begin{aligned}
      \lambda^* x .\text{ let } & [a'_\gamma, a_\gamma] = x, \\
      &[a , b] = a_m \cdot a_\gamma \text{ in }\\
      & E(b, a_{T_1}, a_{T_2}) \cdot [ a'_\gamma , a]      
    \end{aligned}$\\
    \\
    \emph{assuming $a_m \cdot a_\gamma \vDash M$,  $a_{T_1} \cdot [a'_\gamma, !_\rho a_a] \vDash T_1$, $a_{T_2} \cdot [a'_\gamma, !_\pi a_b] \vDash T_2$, then we want to find $a_\bcase$, s.t.\ $a_\bcase\cdot [a'_\gamma, a_\gamma] \vDash \bcase(M, T_1, T_2)$.\\ 
      \subitem if $a_m \cdot a_\gamma = [!_\rho a_a , \ulcorner true \urcorner]$, then $a_\bcase \cdot [a_\gamma , a'_\gamma] \leadsto E(\ulcorner true \urcorner, a_{T_1}, a_{T_2})\cdot [a'_
      \gamma, !_\rho a_a] \leadsto a_{T_1} \cdot [a'_\gamma, !_\rho a_a]$
      \subitem if $a_m \cdot a_\gamma = [!_\pi a_b, \ulcorner false \urcorner]$, then $\dots$} 
  \end{itemize}
\end{proof}

\begin{claim}
  There is a bijection:
  $$RTm(\Gamma, \Pi (x \of{\tau} \rho A \oplus \pi B)\, C) \cong RTm(\Gamma, \Pi (y \of{\tau\rho}A)\, C[\binl y /x]) \times RTm(\Gamma, \Pi(z \of{\tau\pi} B)\, C[\binl z /x]))$$
  \textit{(natural in $\Gamma$).}
\end{claim}

\begin{proof}
  Given a term $\Gamma \vdash f \of{1} (x \of \tau \rho A \oplus \pi B) \to C$, we can derive another term\\ $\Gamma \vdash f^l \of{1} ( y \of {\tau \rho} A) \to C[\binl y /x]$:
  $$
  \infer[Lam]{\Gamma \vdash \lambda y \of{\tau\rho} A .\, f (\binl y) : (y \of {\tau \rho} A) \to C[\binl y / x]}{
    \infer[App]{\Gamma , y \of{\tau\rho} A \vdash f(\binl y) \of{1} C[\binl y /x] }{
      \infer[Weak]{\Gamma , y \of{0} A \vdash f \of{1} (x \of\tau \rho A \oplus \pi B) \to C}{
        \Gamma \vdash f : (x \of \tau \rho A \oplus \pi B) \to C
      }
      &
      \infer[inl]{0\Gamma, y \of\rho A \vdash \binl y \of{1}\rho A \oplus \pi B}{
        \infer[var]{0\Gamma, y \of{1} \vdash y \of{1} A}{
          \vdash 0\Gamma, y \of{1} A
        }
      }
    }
  }
  $$
  Analogously, we can obtain $\Gamma \vdash f^r \of{1} (z \of{\tau\pi} B) \to C[\binr y /x]$.\\
  Now suppose we have terms $\Gamma \vdash l \of{1} (y \of{\tau \rho} A) \to C[\binl y /x]$ and $\Gamma \vdash r \of{1} (z \of{\tau \pi} B) \to C[\binr z / x]$. Using the isomorphism $\Lambda^\mathcal{L}$, we get judgements $ \Gamma, y \of{\tau \rho} A \vdash l^* : C[\binl y / x]$ and\\ $\Gamma , z \of {\tau \pi} B \vdash r^* : C[\binr z /x]$.
\end{proof}


%$case(\mathbf{inl}(x), T_a, T_b) = T_a$
%\\$\Gamma \define x \of{\rho} A$, $\Gamma' \define x \of{0} A$.
%WLOG, let's verify the soundness of $\mathbf{inl}$ and $\mathbf{case}$.\\
%Given a realiser $a_M$ for $\forall \gamma \in |\Gamma|. M(\gamma)$, let $a_{\mathbf{inl}} \define \lambda^* a_\gamma. [\ulcorner true \urcorner, a_M \cdot a_\gamma]$
%Given realisers $a_M$, $a_T$ and $b_T$, s.t:\\
%$\forall \gamma \in |\Gamma|, a_\gamma \in \algA. a_\gamma \vDash_{\Gamma} y \implies a_M \cdot a_y \vDash_{A \oplus B (\gamma)} M(\gamma)$,\\
%$\forall \gamma' \in |\Gamma'|, s \in |A(\gamma')|,  a'_\gamma \in \algA, a_s \in \algA. a'_\gamma \vDash_{\Gamma} \gamma' \land a_s \vDash_{A(\gamma)} s \implies a_T \cdot [a'_\gamma , !_\rho a_s] \vDash_{C(\gamma' , inl(s))} T_a(\gamma', s)$,\\
%$\forall \gamma' \in |\Gamma'|, t \in |B(\gamma')|,  a'_\gamma \in \algA, a_t \in \algA. a'_\gamma \vDash_{\Gamma} \gamma' \land a_t \vDash_{B(\gamma)} t \implies b_T \cdot [a'_\gamma , !_\pi a_t] \vDash_{C(\gamma' , inr(t))} T_b(\gamma', t)$,\\
%let $a_{case} \define
%%\begin{aligned}
%% W_{1 \langle\rho , \pi \rangle} \cdot (\lambda^* & a'_{!\gamma}\ a_{!\gamma}.  &&\comment{-- unwrap $(1  + \langle \rho , \pi \rangle) \Gamma$}\\
%% &\text{let }[inj, a_m] = a_M \text{ in } &&\comment{-- pattern match on $M$}\\
%% &\text{let }[collapsed, a_r] = a_{!\gamma} \text{ in } &&\comment{-- check if $\langle \pi , \rho \rangle$ collapsed}\\
%% &\text{let }[c_T, inj'] = E([a_T , \ulcorner true \urcorner], [b_T , \ulcorner false \urcorner]) \cdot inj \text{ in } &&\comment{--}\\
%% &\comment{-- check the right realiser and rewrap a boolean value}&&\\
%% &\comment{-- so the linearity of inj is not violated }&&\\
%% & c_T \cdot [D\cdot a'_{!\gamma} ,  &&\\
%% & \quad\quad E(\lambda^*r\, m. , \lambda^*r . , ) \cdot collapsed \cdot a_r \cdot a_m]
%%\end{aligned}$
%\begin{aligned}
%  W_{1 \langle\rho , \pi \rangle} \cdot (\lambda^* & a'_{!\gamma}\ a_{!\gamma}.  &&\comment{-- unwrap $(1  + \langle \rho , \pi \rangle) \Gamma$}\\
%  & \dots F_{\langle \pi , \rho \rangle} \cdot !_{\langle \pi , \rho \rangle}(\lambda^* r. a_M \cdot r ) . a_{!\gamma} \dots &&
%\end{aligned}$\\
%Problem : We end up with something of the shape $!_{\langle \pi , \rho \rangle}([\ulcorner boolean \urcorner,  x ])$ and want to select the projection depending on the inner boolean value.
Now we can focus on the relational part 
\begin{align*}
  (\rho A \oplus \pi B)(\gamma)_R &\define A(\gamma)_R \sqcup B(\gamma)_R\\
  (\rho A \oplus \pi B)(\gamma)_{refl} &\define A(\gamma)_{refl} \sqcup B(\gamma)_{refl}\\
  (\rho A \oplus \pi B)(\gamma)_{\sigma} &\define A(\gamma)_{\sigma} \sqcup B(\gamma)_{\sigma}, \quad \sigma \in \{\text{src}, \text{tgt}\}
\end{align*}
Suppose $A,B$ are discrete. Then $\rho A \oplus \pi B$ is also discrete.
(Since coproducts preserve isos).


\end{document}
