\usepackage{amssymb}
\usepackage{amsmath}
\usepackage{amsthm}
\usepackage{xspace}
\usepackage{tikz}
\usepackage{tikz-cd}
\usepackage{enumitem}
\usepackage{adjustbox}
\usepackage{stmaryrd}
\usepackage{mathrsfs}  
\usepackage{proof}
\usepackage[left=5em, right=5em]{geometry}

%%%%%%%%%%%%%%%%%%%%%%%%%%%%%%%%%%%% Link colors %%%%%%%%%%%%%%%%%%%%%%%%%%%%%%%%%%%
\usepackage{hyperref}
\usepackage[capitalize,nameinlink]{cleveref}

\hypersetup{
  colorlinks,
  citecolor=black,
  filecolor=black,
  linkcolor=black,
  urlcolor=black,
  linktoc=all
}

%%%%%%%%%%%%%%%%%%%%%%%%%%%%%%%%%%% Bibliography %%%%%%%%%%%%%%%%%%%%%%%%%%%%%%%%%%%

\usepackage[backend=biber, isbn=false, doi=false, style=numeric, 
citestyle=numeric, giveninits=true]{biblatex}
\bibliography{lib.bib}

%%%%%%%%%%%%%%%%%%%%%%%%%%%%%%%% Theorem enivornment %%%%%%%%%%%%%%%%%%%%%%%%%%%%%%%

%use continious uniform numbering

\newtheorem{thm}{Theorem}[section]
\newtheorem{prop}[thm]{Proposition}
\newtheorem{cor}[thm]{Corollary}
\newtheorem{lemma}[thm]{Lemma}
\newtheorem*{claim}{Claim}
\newtheorem*{fact}{Fact}

\theoremstyle{definition}
\newtheorem{remark}[thm]{Remark}
\newtheorem{definition}[thm]{Definition}
\newtheorem{example}[thm]{Example}

%% remove section prefix
\renewcommand{\thethm}{\arabic{thm}}

% \makeatletter
%\def\th@definition{%
%  \thm@notefont{\normalfont\itshape}
%}
%\makeatother


%%%%%%%%%%%%%%%%%%%%%%%%%%%%%%%%%%%% Always math %%%%%%%%%%%%%%%%%%%%%%%%%%%%%%%%%%%

\makeatletter
\let\ea\expandafter

\def\mdef#1#2{\ea\ea\ea\gdef\ea\ea\noexpand#1\ea{\ea\ensuremath\ea{#2}\xspace}}
\def\alwaysmath#1{\ea\ea\ea\global\ea\ea\ea\let\ea\ea\csname your@#1\endcsname\csname #1\endcsname
  \ea\def\csname #1\endcsname{\ensuremath{\csname your@#1\endcsname}\xspace}}

\makeatother

%%%%%%%%%%%%%%%%%%%%%%%%%%%%%%%% Letter decorations %%%%%%%%%%%%%%%%%%%%%%%%%%%%%%%%

\makeatletter

\newcount\foreachcount

\def\foreachletter#1#2#3{\foreachcount=#1
  \ea\loop\ea\ea\ea#3\@alph\foreachcount
  \advance\foreachcount by 1
  \ifnum\foreachcount<#2\repeat}

\def\foreachLetter#1#2#3{\foreachcount=#1
  \ea\loop\ea\ea\ea#3\@Alph\foreachcount
  \advance\foreachcount by 1
  \ifnum\foreachcount<#2\repeat}

% Particular commands for typefaces, sometimes with the first letter
% different.
\def\autofmt@n#1\autofmt@end{\mathrm{#1}}
\def\autofmt@b#1\autofmt@end{\mathbf{#1}}
\def\autofmt@d#1#2\autofmt@end{\mathbb{#1}\mathsf{#2}}
\def\autofmt@c#1#2\autofmt@end{\mathcal{#1}\mathit{#2}}
\def\autofmt@s#1#2\autofmt@end{\mathscr{#1}\mathit{#2}}
\def\autofmt@f#1\autofmt@end{\mathsf{#1}}
%\def\autofmt@k#1\autofmt@end{\mathfrak{#1}}
%% Particular commands for decorations.
\def\autofmt@u#1\autofmt@end{\underline{\smash{\mathsf{#1}}}}
\def\autofmt@U#1\autofmt@end{\underline{\underline{\smash{\mathsf{#1}}}}}
\def\autofmt@h#1\autofmt@end{\widehat{#1}}
\def\autofmt@r#1\autofmt@end{\overline{#1}}
\def\autofmt@t#1\autofmt@end{\widetilde{#1}}
\def\autofmt@k#1\autofmt@end{\check{#1}}

% Defining multi-letter commands.  Use this like so:
% \autodefs{\bSet\cCat\cCAT\kBicat\lProf}
\def\auto@drop#1{}
\def\autodef#1{\ea\ea\ea\@autodef\ea\ea\ea#1\ea\auto@drop\string#1\autodef@end}
\def\@autodef#1#2#3\autodef@end{%
  \ea\def\ea#1\ea{\ea\ensuremath\ea{\csname autofmt@#2\endcsname#3\autofmt@end}\xspace}}
\def\autodefs@end{blarg!}
\def\autodefs#1{\@autodefs#1\autodefs@end}
\def\@autodefs#1{\ifx#1\autodefs@end%
  \def\autodefs@next{}%
  \else%
  \def\autodefs@next{\autodef#1\@autodefs}%
  \fi\autodefs@next}

\makeatother

%%%%%%%%%%%%%%%%%%%%%%%%%%%%%%%%%%% Miscellaneous %%%%%%%%%%%%%%%%%%%%%%%%%%%%%%%%%%


\def\define{:=}

\let\comp\circ

\def\pow{\mathcal{P}}
\newcommand{\carr}[1]{\ensuremath{|#1|}}

\newcommand{\ob}[1]{\ensuremath{Ob(#1)}}

\renewcommand{\phi}{\varphi}
\alwaysmath{Gamma}

\setlength{\parindent}{0pt}

\newcommand{\terminal}[1]{\mathbf{1_{#1}}}


%%%%%%%%%%%%%%%%%%%%%%%%%%%%%%%%%%% Comments %%%%%%%%%%%%%%%%%%%%%%%%%%%%%%%%%%%%%%%

\newcommand{\FNF}[1]{\textcolor{blue}{\textbf{FNF:} #1}}
\newcommand{\GN}[1]{\textcolor{magenta}{\textbf{GN:} #1}}